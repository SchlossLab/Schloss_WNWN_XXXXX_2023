% Options for packages loaded elsewhere
\PassOptionsToPackage{unicode}{hyperref}
\PassOptionsToPackage{hyphens}{url}
%
\documentclass[
]{article}
\usepackage{amsmath,amssymb}
\usepackage{lmodern}
\usepackage{iftex}
\ifPDFTeX
  \usepackage[T1]{fontenc}
  \usepackage[utf8]{inputenc}
  \usepackage{textcomp} % provide euro and other symbols
\else % if luatex or xetex
  \usepackage{unicode-math}
  \defaultfontfeatures{Scale=MatchLowercase}
  \defaultfontfeatures[\rmfamily]{Ligatures=TeX,Scale=1}
\fi
% Use upquote if available, for straight quotes in verbatim environments
\IfFileExists{upquote.sty}{\usepackage{upquote}}{}
\IfFileExists{microtype.sty}{% use microtype if available
  \usepackage[]{microtype}
  \UseMicrotypeSet[protrusion]{basicmath} % disable protrusion for tt fonts
}{}
\makeatletter
\@ifundefined{KOMAClassName}{% if non-KOMA class
  \IfFileExists{parskip.sty}{%
    \usepackage{parskip}
  }{% else
    \setlength{\parindent}{0pt}
    \setlength{\parskip}{6pt plus 2pt minus 1pt}}
}{% if KOMA class
  \KOMAoptions{parskip=half}}
\makeatother
\usepackage{xcolor}
\usepackage[margin=1.0in]{geometry}
\usepackage{graphicx}
\makeatletter
\def\maxwidth{\ifdim\Gin@nat@width>\linewidth\linewidth\else\Gin@nat@width\fi}
\def\maxheight{\ifdim\Gin@nat@height>\textheight\textheight\else\Gin@nat@height\fi}
\makeatother
% Scale images if necessary, so that they will not overflow the page
% margins by default, and it is still possible to overwrite the defaults
% using explicit options in \includegraphics[width, height, ...]{}
\setkeys{Gin}{width=\maxwidth,height=\maxheight,keepaspectratio}
% Set default figure placement to htbp
\makeatletter
\def\fps@figure{htbp}
\makeatother
\setlength{\emergencystretch}{3em} % prevent overfull lines
\providecommand{\tightlist}{%
  \setlength{\itemsep}{0pt}\setlength{\parskip}{0pt}}
\setcounter{secnumdepth}{-\maxdimen} % remove section numbering
\newlength{\cslhangindent}
\setlength{\cslhangindent}{1.5em}
\newlength{\csllabelwidth}
\setlength{\csllabelwidth}{3em}
\newlength{\cslentryspacingunit} % times entry-spacing
\setlength{\cslentryspacingunit}{\parskip}
\newenvironment{CSLReferences}[2] % #1 hanging-ident, #2 entry spacing
 {% don't indent paragraphs
  \setlength{\parindent}{0pt}
  % turn on hanging indent if param 1 is 1
  \ifodd #1
  \let\oldpar\par
  \def\par{\hangindent=\cslhangindent\oldpar}
  \fi
  % set entry spacing
  \setlength{\parskip}{#2\cslentryspacingunit}
 }%
 {}
\usepackage{calc}
\newcommand{\CSLBlock}[1]{#1\hfill\break}
\newcommand{\CSLLeftMargin}[1]{\parbox[t]{\csllabelwidth}{#1}}
\newcommand{\CSLRightInline}[1]{\parbox[t]{\linewidth - \csllabelwidth}{#1}\break}
\newcommand{\CSLIndent}[1]{\hspace{\cslhangindent}#1}
\usepackage{upgreek}
\usepackage{booktabs}
\usepackage{longtable}
\usepackage{graphicx}
\usepackage{array}
\usepackage{multirow}
\usepackage{wrapfig}
\usepackage{float}
\usepackage{colortbl}
\usepackage{pdflscape}
\usepackage{tabu}
\usepackage{threeparttable}
\usepackage{threeparttablex}
\usepackage[normalem]{ulem}
\usepackage{makecell}
\usepackage{setspace}
\doublespacing
\usepackage[left]{lineno}
\linenumbers
\modulolinenumbers
\usepackage{helvet} % Helvetica font
\renewcommand*\familydefault{\sfdefault} % Use the sans serif version of the font
\usepackage[T1]{fontenc}
\usepackage[shortcuts]{extdash}
\ifLuaTeX
  \usepackage{selnolig}  % disable illegal ligatures
\fi
\IfFileExists{bookmark.sty}{\usepackage{bookmark}}{\usepackage{hyperref}}
\IfFileExists{xurl.sty}{\usepackage{xurl}}{} % add URL line breaks if available
\urlstyle{same} % disable monospaced font for URLs
\hypersetup{
  hidelinks,
  pdfcreator={LaTeX via pandoc}}

\author{}
\date{\vspace{-2.5em}}

\begin{document}

\hypertarget{waste-not-want-not-revisiting-the-analysis-that-called-the-practice-of-rarefying-microbiome-data-into-question}{%
\section{Waste not, want not: Revisiting the analysis that called the
practice of rarefying microbiome data into
question}\label{waste-not-want-not-revisiting-the-analysis-that-called-the-practice-of-rarefying-microbiome-data-into-question}}

\vspace{20mm}

\textbf{Running title:} Review of ``Waste not, want not''

\vspace{20mm}

Patrick D. Schloss\({^\dagger}\)

\vspace{40mm}

\({\dagger}\) To whom corresponsdence should be addressed:

\href{mailto:pschloss@umich.edu}{pschloss@umich.edu}

Department of Microbiology \& Immunology

University of Michigan

Ann Arbor, MI 48109

\vspace{20mm}

\textbf{Research article}

\newpage

\hypertarget{abstract}{%
\subsection{Abstract}\label{abstract}}

\newpage

\hypertarget{introduction}{%
\subsection{Introduction}\label{introduction}}

Since the development of highly parallelized sequencing technologies
that enable microbiome researchers to distribute sequences across
multiple samples in a single sequencing run, researchers have struggled
to produce a consistent number of sequences from each sample in a
dataset. It is common to observe more than 10-fold variation in the
number of sequences per sample {[}XXXXX{]}. Researchers desire
strategies to limit uneven sampling effort and need methods to control
for he uneveness in their analyses. Of course, uneven sampling is not
unique to microbiome research and is a challenge faced by all community
ecologists. Common approaches to controlling uneven sampling efforts
include use of proportional abundance (i.e., relative abundance),
normalization of counts, parameter estimation, and rarefaction.

In 2014 Paul McMurdie and Susan Holmes published ``Waste not, want not:
why rarefying microbiome data is inadmissible'' (WNWN) in PLOS
Computational Biology {[}XXXXX{]}. This paper has had a significant
impact on the approaches that microbiome researchers use to analyze 16S
rRNA gene sequence data. According to Google Scholar, this paper has
been cited more than 2,400 times as of April 2023. Anecdotely, I have
received correspondence from researchers over the past 10 years asking
how to address critiques from reviewers who criticize my correspondents'
analysis for using rarefaction (e.g., see
\href{https://twitter.com/inanna_nalytica/status/1264299305067786242}{this
Twitter thread}). I have also received similar comments from reviewers
regarding my own work. Most recently, I received the critique for a
preprint that I posted in 20XX that analyzes the practice of removing
rare taxa from microbiome analyses {[}XXXXX{]}. In the process of
responding to these reviewers' comments and preparing a manuscript
investigating rarefaction and other approaches to control for uneven
sequencing effort, I decided to reassess the WNWN study including their
definitions, simulations, and analyses.

\hypertarget{confusion-regarding-what-is-meant-by-rarefying-and-rarefaction}{%
\subsubsection{Confusion regarding what is meant by ``rarefying'' and
``rarefaction''}\label{confusion-regarding-what-is-meant-by-rarefying-and-rarefaction}}

As I attempted to reproduce the results of WNWN, I noticed that the step
that rarefied the data only performed one subsampling of the data. This
caused me to re-inspect how McMurdie and Holmes defined ``rarefying'' in
the following quoted text from their paper:

\begin{quote}
Instead, microbiome analysis workflows often begin with an ad hoc
library size normalization by random subsampling without replacement, or
so-called rarefying {[}17{]}--{[}19{]}. There is confusion in the
literature regarding terminology, and sometimes this normalization
approach is conflated with a non-parametric resampling technique ---
called rarefaction {[}20{]}, or individual-based taxon re-sampling
curves {[}21{]} --- that can be justified for coverage analysis or
species richness estimation in some settings {[}21{]}, though in other
settings it can perform worse than parametric methods {[}22{]}. Here we
emphasize the distinction between taxon re-sampling curves and
normalization by strictly adhering to the terms rarefying or rarefied
counts when referring to the normalization procedure, respecting the
original definition for rarefaction. Rarefying is most often defined by
the following steps {[}18{]}.

\begin{enumerate}
\def\labelenumi{\arabic{enumi}.}
\tightlist
\item
  Select a minimum library size, N\textsubscript{L,m}. This has also
  been called the rarefaction level {[}17{]}, though we will not use the
  term here.
\item
  Discard libraries (microbiome samples) that have fewer reads than
  N\textsubscript{L,m}.
\item
  Subsample the remaining libraries without replacement such that they
  all have size N\textsubscript{L,m}.
\end{enumerate}

Often N\textsubscript{L,m} is chosen to be equal to the size of the
smallest library that is not considered defective, and the process of
identifying defective samples comes with a risk of subjectivity and
bias. In many cases researchers have also failed to repeat the random
subsampling step (3) or record the pseudorandom number generation
seed/process --- both of which are essential for reproducibility.
\end{quote}

It was unfortunate that McMurdie and Holmes used the term ``rarefying''
here and throughout their manuscript. The authors were correct to state
that the distinction between ``rarefying'' and ``rarefaction'' is
confusing and leads to their conflation. Adding to the confusion is that
the papers cited in the first sentence of this quote (i.e., their
references {[}17{]}--{[}19{]}) either do not use the words ``rarefy'' or
``rarefying'' or use them interchangably with ``rarefaction''. In my
experience, subsequent researchers have conflated the results of this
study of the effects of rarefying data with rarefaction of data. As an
example, Willis (XXXXX) describes problems with rarefaction rather than
rarefying data when citing WNWN in her paper proposing alternatives to
rarefaction for use with alpha diversity data:

\begin{quote}
Unfortunately, rarefaction is neither justifiable nor necessary, a view
framed statistically by McMurdie and Holmes (2014) in the context of
comparison of relative abundances.
\end{quote}

In hindsight, as shown in the quoted text from WNWN, McMurdie and Holmes
did emphasize the distinction between rarefying and rarefaction.
However, because they seem to have coined a new meaning for rarefying,
they added to the confusion by using the generally used verb form of
rarefaction. Further confusion comes from the author's admonition in the
final sentence that some researchers have failed to repeat the
subsampling step. To most scientists, repeating the subsampling step is
rarefaction. My preference is to use \emph{subsampling} as the term
describing the process they refer to as rarefying. In other words,
subsampling is rarefaction, but with a single randomization. To minimize
confusion, I will use subsamping in place of rarefying.

I propose the following definition of rarefaction:

\begin{enumerate}
\def\labelenumi{\arabic{enumi}.}
\tightlist
\item
  Select a minimum library size, N\textsubscript{L,m}. Researchers are
  encouraged to report the value of N\textsubscript{L,m}.
\item
  Discard samples that have fewer reads than N\textsubscript{L,m}.
\item
  Subsample the remaining libraries without replacement such that they
  all have size N\textsubscript{L,m}.
\item
  Compute the desired metric (e.g., richness, Shannon diversity,
  Bray-Curtis distances) using the subsampled data
\item
  Repeat steps 3 and 4 a large number of iterations (e.g, 100 or 1,000).
  Researchers are encouraged to report the number of iterations.
\item
  Compute summary statistics using values generated from the subsampled
  data
\end{enumerate}

This definition aligns well with how rarefaction was originally defined
for comparing richness (i.e., the number of taxa in a community) across
communities when communities are sampled to different depths
{[}XXXXXXXXX{]}. With this more general approach to rarefaction,
rarefaction can be performed using any alpha or beta diversity metric.
This strategy has been widely used by my research group and others and
is available in the mothur software package using commands such as
\texttt{summary.single}, \texttt{rarefaction.single},
\texttt{dist.shared}. The procedure outlined above could also be used
for hypothesis tests of differential abundance; however, thought would
need to be given to how to synthesize the results of these tests across
a large number of replications.

\hypertarget{description-of-simulation-a-from-wnwn}{%
\subsubsection{Description of ``Simulation A'' from
WNWN}\label{description-of-simulation-a-from-wnwn}}

McMurdie and Holmes analyzed the effect of rarefying and other
approaches on clustering accuracy using what they called ``Simulation
A'' in their Figure 2A and elsewhere in their paper. In Simulation A,
they investigated the ability to correctly assign simulated microbiome
samples to one of two clusters representing two simulated treatment
groups. Thankfully, the R code for Simulation A was provided by the
authors in the
\texttt{simulation-cluster-accuracy/simulation-cluster-accuracy-server.Rmd}
R-flavored markdown file that was published as Protocol S1 in the
original paper. I will outline the simulation strategy below and will
reference line numbers from their R-flavored markdown file with ``L'' as
a prefix.

For the WNWN cluster analysis, sampling distributions for the two
treatment groups were generated using human fecal and ocean data
originally take from the GlobalPaterns dataset (L129) {[}XXXXX{]}. To
generate a fecal and ocean distribution, the authors included any
operational taxonomic unit (OTU) that appeared in more than one of the 4
fecal and 3 ocean samples (L60 and L137). The OTUs were sorted by how
many of the 7 samples the OTUs were observed in followed by their total
abundance across all 7 samples (L139). From this sorted list they
identified the identifiers of the first 2000 OTUs (L66). Returning to
the 7 samples they selected the data for the corresponding 2000 OTUs and
pooled the OTU abundances of the fecal and ocean samples separately to
create two sampling distributions (L144, L159-160, L197-198). Next, the
fecal and ocean distributions were mixed in 8 different fractions to
generate two community types that differend by varying effect sizes
(i.e., 1, 1.15, 1.25, 1.5, 1.75, 2, 2.5, and 3.5; L170-195, L220); an
effect size of 1 generated a null model with no difference between the
treatment groups. To simulate the variation in sequencing depth across
the 80 samples, they normalized he number of sequences from each of the
26 samples in the GlobalPatterns dataset so that the median number of
sequences (Ñ\textsubscript{L}) for the GlobalPatterns had 1,000, 2,000,
5,000, or 10,000 sequences (L324-325). They then randomly sampled the 26
normalized sequencing depths to generate 80 sampling depths. From each
community type, they simulated 40 samples by sampling to the desired
number of sequences (L73, L230-233 and L326-327). Each simulation
condition was repeated 5 times (L85). This resulted in 160 simulations
(8 effect sizes x 4 median sampling depths x 5 replicates = 160
simulations). Finally, they removed rare and low prevalence OTUs in two
steps. First, they removed any OTUs whose total abundance was less than
3 across all 80 samples and that did not appear in at least 3 samples
(L368-386). Second, they removed any OTUs that did not have more than 1
sequence in more than 5\% of the 80 samples (i.e., 4 samples) and that
did not have a total abundance across the 80 samples greater than one
half of the number of samples in each community type (i.e., 20)
(L523-538, L551).

After generating the OTU counts for the 160 simulated communities, the
authors applied several normalization methods, distance calculations,
and clustering algorithms to the data. To normalize the OTU data the
original analysis either applied no normalization procedure, calculated
OTU relative abundances, subsampled the data, performed variance
stabilization using the DESeq R package, or performed Upper-Quartile
log-fold change normalization using the edgeR R package. In subsampling
the data, the authors either included all of the samples or removed
samples whose sequencing depth fell below the 5, 10, 15, 20, 25, and 40
percentile across all 80 samples. Subsampled data were then used as
input to calculate distances between samples using Bray-Curtis,
Euclidean, Unweighted UniFrac, and Weighted Unifrac distances as
implemented in the Phyloseq R package, Poisson distance as implemented
in the PoiClaClu R package, and top-mean squared difference as
implemented in the edgeR R package. Un-normalized data were used to
calculate Bray-Curtis, Euclidean, Poisson, and Weighted UniFrac
distances. Relative abundance data were used to calculate Bray-Curtis,
Unweigthed UniFrac, and Weighted UniFrac distances. DESeq variance
stabilization normalized data were used to calculate Bray-Curtis,
Euclidean, and Weighted Unifrac distances. The Upper-Quartile log-fold
change normalized data were only used to calculate top-mean squared
difference distances. The resulting distance matrices were used to
cluster the 80 samples into one of two clusters using partitioning
around the medoid (PAM), K-means clustering, and hierarchical
clustering. Although data for all three methods were presented in the
supplementary Protocol S1, only the PAM data are presented in the main
manuscript. The accuracy of the clustering assignments was quantified as
the fraction of the 80 samples that were assigned to the correct
cluster. Because some of the subsampling conditions removed samples
those were counted as mis-clustered samples yielding minimum accuracies
below 50\%.

\hypertarget{critique-of-the-original-simulation-design}{%
\subsubsection{Critique of the original simulation
design}\label{critique-of-the-original-simulation-design}}

Although all simulations represent an artificial representation of
reality and can be critiqued, eleven elements of the design of
Simulation A warrant further review.

\begin{enumerate}
\def\labelenumi{\arabic{enumi}.}
\tightlist
\item
  Simulated conditions were only replicated 5 times each
\item
  The average sizes of the libraries were small by modern standards
\item
  DESeq-based variance stabilization was used with distance calculation
  methods that are sensitive to negative values
\item
  A single subsampling of each dataset was evaluated rather than using
  rarefaction, which likely resulted in noisier data
\item
  Results using PAM clustering were not directly compared to those of
  K-means and hierarchical clustering
\item
  Subsampling removed the smallest 15\% of the samples, which penalized
  accuracy values by 15 percentage points
\item
  The distribution of library sizes was not typical of those commonly
  seen in microbiome analyses
\item
  A filtering step was applied to remove rare taxa from the simulated
  datasets
\item
  No accounting for difference in performance when library sizes are
  confounded with treatment group
\item
  Clustering accuracy was used rather than direct comparisons of beta
  diversity
\item
  There was no consideration of effects of normalization methods on
  alpha diversity metrics
\end{enumerate}

These points will serve as an outline for the Results section. After
replicating the original simulations, these points will be evaluated to
reassess whether subsampling or rarefaction are ``inadmissible''.

\hypertarget{results}{%
\subsection{Results}\label{results}}

\hypertarget{replication-of-wnwn-simulations-and-results}{%
\subsubsection{Replication of WNWN simulations and
results}\label{replication-of-wnwn-simulations-and-results}}

Before assessing the impact of the points I critiqued above, I attempted
to replicate the results shown in Figures 4 and 5 of the original paper
using the authors' original code. I created a Conda environment that
used the R version and package versions that were as close as possible
to those used in the original paper. Because of the slight differences
in packages, it was necessary to apply several patches to the original
R-flavored markdown file to render document. I was able to generate a
figure similar to that presented as Figures 4 and 5 of the original
paper. My results are shown in Figures S2
(\textbf{norarefy-source/simulation-cluster-accuracy/Figure\_3.pdf}) and
S3 (\textbf{norarefy-source/simulation-cluster-accuracy/Figure\_4.pdf})
of this paper, respectively. The differences in results are likely due
to differences in software versions and operating systems. It is also
worth noting that the published versions of the two figures differ from
those included in Protocol S1 within the rendered html file
(\texttt{simulation-cluster-accuracy/simulation-cluster-accuracy-server.html})
and that the figure numbers are one higher in the paper than those
generated by the R-flavored markdown file (i.e., Protocol S1 has Figures
3 and 4 corresponding to the published Figures 4 and 5). Regardless of
the differences, my results were qualitatively similar to the originals.

\hypertarget{simulated-conditions-were-only-replicated-5-times-each}{%
\subsubsection{1. Simulated conditions were only replicated 5 times
each}\label{simulated-conditions-were-only-replicated-5-times-each}}

Each simulated condition was replicated 5 times in WNWN and the paper
reports the mean and standard deviation of the replicate clustering
accuracies. The relatively small number of replicates accounts for the
jerkiness of the lines in the original Figures 4 and 5 (e.g.~the
Bray-Curtis distances calculated on the DESeqVS normalized data). A
better approach would have been to use 100 replicates as this would
reduce the dependency of the results on the random number generator's
seed. By increasing the number of replicates it was also possible to
compare the probability of falsely and correctly clustering samples from
the same and different treatment groups together (see points 9-11,
below). Because the accuracies were not symmetric around the mean
accuracy values the median and 95\% confidence intervals or
intraquartile range should have been reported. To test the effect of
increasing the number of replicates, I pulled apart the code in
\texttt{simulation-cluster-accuracy/simulation-cluster-accuracy-server.Rmd}
into individual R and bash scripts that were executed using a Snakemake
workflow with the same Conda environment I used above. This was
necessary since the number of simulated conditions increased 20-fold
with the additional replicates. Such intense data processing was not
practical within a single R-flavored markdown document. Again, the
observed results were qualitiatively similar to those generated using
the single R-flavored markdown file (Figures
\textbf{pam\_subsample15\_fig\_4.pdf} and
\textbf{pam\_subsample\_fig\_5.pdf}). The increased number of
replications resulted in smoother lines and allowed me to present
empirical 95\% confidence intervals. For all analyses in the remainder
of this paper, I used 100 randomized replicates per condition.

\hypertarget{the-average-sizes-of-the-libraries-were-small-by-modern-standards}{%
\subsubsection{2. The average sizes of the libraries were small by
modern
standards}\label{the-average-sizes-of-the-libraries-were-small-by-modern-standards}}

In the 10 years since WNWN was published, sequencing technology has
advanced and sequence collections have grown considerably. For more
modern datasets, it would be reasonable to expect a median number of
sequences larger than 10,000 (see Table 1 of {[}Singleton Paper
XXXXXXXX{]}). Therefore, I included an additional median depth of
sampling value of 50,000 sequences with the original four median
sequencing sampling depths (i.e., 1,000, 2,000, 5,000, 10,000).
Additional sequencing coverage would be expected to result in more
robust distance values since there would be more information represented
in the data. Indeed, the added sampling depth showed higher accuracy
values at lower effect sizes for the combinations of normalization
methods and distance calculations (Figure S4
(\textbf{pam\_subsample15\_fig\_4.pdf})). Increased sequencing coverage
also resulted in improved clustering accuracy for lower effect sizes
when the library size minimum quantile was decreased (Figure S5
(\textbf{pam\_subsample\_fig\_5.pdf})). I will revisit the choice of the
library size minimum quantile below.

\hypertarget{deseq-based-variance-stabilization-was-used-with-distance-calculation-methods-that-are-sensitive-to-negative-values}{%
\subsubsection{3. DESeq-based variance stabilization was used with
distance calculation methods that are sensitive to negative
values}\label{deseq-based-variance-stabilization-was-used-with-distance-calculation-methods-that-are-sensitive-to-negative-values}}

Close comparison of the original Figure 4 and my version
(\textbf{pam\_subsample15\_fig\_4.pdf}) revealed one important
difference between the two plots. In the original analysis, the
accuracies for the Weighted UniFrac distances at the largest effect size
(i.e., 3.5) were 1.0 for median sequencing depths of 1,000, 2,000, and
10,000. In my version of the analysis, the values for the same
sequencing depths were 0.88, 0.89, and 1.00, respectively. The 95\%
confidence interval for these accuracies spaned between 0.51 and 1.00.
The Bray-Curtis distances were also different by both methods at smaller
effect sizes and had wide confidence intervals. Inspection of the DESeq
normalized OTU counts revealed that the method resulted in negative
values. In fact, rendering the R-flavored markdown files in WNWN's
Protocol S1 generated warning messages when passing the DESeq normalized
counts to the Bray-Curtis calculator, which said, ``results may be
meaningless because data have negative entries in method `bray'\,''.
Although the Weighted UniFrac calculator function did not generate a
similar warning message, negative values would also result in similarly
meaningless distances. Both are due to the fact that the calculators sum
the counts of each OTU in both samples being compared. In contrast, a
Euclidean distance does not use a similar sum, but sums the square of
the difference between the OTU abundance in each sample. To assess the
prevalence of negative counts in the simulated data, I quantified the
fraction of negative values in the OTU matrix from each simulation and
counted the number of simulations where the normalized OTU table had at
least one negative value (Figure \textbf{deseq\_negative\_value.pdf}).
In general the fraction of negative OTU counts increased with effect
size, but decreased with sequencing effort. The fraction of simulations
with at least onee negative value increased with effect size and
sequencing effort. The high frequency of negative OTU counts resulted in
highly variable Bray-Curtis and Weighted UniFrac values. It is likely
that because the WNWN analysis only used 5 replicates that the large
variation in accuracies at high effect sizes was missed initially. For
the rest of this reanalysis study, I will only report results using the
DESeq-based variance stabilization normalization with the Euclidean
distance.

\hypertarget{a-single-subsampling-of-each-dataset-was-evaluated-rather-than-using-rarefaction-which-likely-resulted-in-noisier-data}{%
\subsubsection{4. A single subsampling of each dataset was evaluated
rather than using rarefaction, which likely resulted in noisier
data}\label{a-single-subsampling-of-each-dataset-was-evaluated-rather-than-using-rarefaction-which-likely-resulted-in-noisier-data}}

As noted above, the original jargon that was used in WNWN was confusing
to many who conflated rarefying/subsampling with rarefaction. A more
robust analysis would have used rarefaction since it would have averaged
across random subsamplings, which individually would be unlikely to
represent the overall composition of the communities. Rather than being
guilty of ``omission of available valid data'' as claimed in WNWN,
rarefaction uses all of the available data. To compare subsampling and
rarefaction, I removed the 15\% of samples with the lowest number of
sequences for each of the 100 simulated datasets and compared the
clustering accuracies from a single subsampling to rarefaction with 100
randomizations. This analysis revealed two benefits of rarefaction.
First, the median distances generated by rarefaction was always at least
as large as those from a single subsample (Figure
\textbf{subsample\_rarefaction\_compare.pdf}). The difference was most
pronounced for smaller average library sizes and at smaller effect
sizes. The Unweighted UniFrac distances were most impacted by the use of
rarefaction over subsampling. Second, the intraquartile ranges for the
distances generated by rarefaction were generally smaller than those by
subsampling and showed similar trends to the difference in the median
distances (Figure \textbf{subsample\_rarefaction\_compare.pdf}). The
intraquartile ranges for Bray-Curtis, Euclidean, and Unweighted UniFrac
distances were actually larger by rarefaction than by subsampling at
small effect sizes and average library sizes; however at larger values
the subsampling intraquartile range was larger than by rarefaction for
these distance calculations. Because rarefaction incorporates more of
the data and generally performed better than subsampling, the remainder
of this analysis will report results using rarefaction rather than by
subsampling, except when noted.

\hypertarget{results-using-pam-clustering-were-not-directly-compared-to-those-of-k-means-and-hierarchical-clustering}{%
\subsubsection{5. Results using PAM clustering were not directly
compared to those of K-means and hierarchical
clustering}\label{results-using-pam-clustering-were-not-directly-compared-to-those-of-k-means-and-hierarchical-clustering}}

The clustering accuracy measurements in the body of the manuscript were
determined using PAM-based clusters while Protocol S1 also includes
K-means and hierarchichal clustering. Although the data were not
displayed in a manner that lent itself to direct comparison in Protocol
S1, close inspection of the rendered figures suggested that PAM may not
have been the optimal choice in all situations. Rather, K-means
clustering appeared to perform better in many simulations. Because the
accuracies were the smallest at lower effect sizes, I focused my
comparison at the effect size of 1.15. For each set of 100 replicated
simulated datasets, I compared the clustering accuracy across clustering
methods to see how often each clustering method resulted in the highest
accuracy (Figure \textbf{compare\_cluster\_methods.pdf}). Indeed,
K-means clustering performed better than the other methods. Among all
combinations of normalization methods, distance calculations, and read
depths, PAM clustering resulted in clustering accuracies as good or
better than the other methods in 49.92\% of the randomizations (Figure
\textbf{compare\_cluster\_methods.pdf}). K-means clustering was at least
as good as the other methods in 74.39\% of the randomizations. HClust
was at least as good as the other methods for 44.32\% of the
randomizations. I specifically compared the clustering accuracies using
rarefaction for each of the distance calculations methods using PAM and
K-means clustering. Among the 30 combinations of distance calculations
and read depths, K-means performed better than PAM in 29 cases with PAM
doing better in the 1 other case (i.e., calculating distances with
Euclidean using 10,000 sequences). When using subsampled data, K-means
clustering performed better than PAM in each case. Because K-means
clustering did so much better than PAM clustering in the simulated
conditions, I will use K-means clustering for the remainder of this
study.

\hypertarget{subsampling-removed-the-smallest-15-of-the-samples-which-penalized-accuracy-values-by-15-percentage-points}{%
\subsubsection{6. Subsampling removed the smallest 15\% of the samples,
which penalized accuracy values by 15 percentage
points}\label{subsampling-removed-the-smallest-15-of-the-samples-which-penalized-accuracy-values-by-15-percentage-points}}

In WNWN, the authors quantified the tradeoff between median sequencing
depth, the number of samples removed below the threshold, and clustering
accuracy (original Figure 5, my Figure S5). Although the optimal
threshold varied by distance metric, normalization method, and
sequencing depth, they removed samples whose number of sequences was
less than the 15th percentile (L404-419). They acknowledged that this
screening step, which was only used with subsampling, would decrease
clustering accuracy putting it at a relative disadvantage to the other
methods (page 5, column 1, last paragraph). Therefore, it was not
surprising that the peak clustering accuracy for their subsampled data
was at 85\%. Because the true best result would not known \emph{a
priori} in an actual microbiome study, it would be impossible for
researchers to conduct a sensitivity analysis comparing the tradeoffs
between sequencing depth, sample number, and clustering accuracy to
select a sampling depth for their analysis. The differences in
clustering accuracy between subsampling and rarefaction and using PAM
and K-means clustering indicated that it was necessary to reassess the
tradeoff between the library size minimum quantile and clustering
accuracy. When using rarefaction, K-means clustering, and only
considering conditions with 2,000 or more sequences, there was not a
condition where setting a higher threshold resulted in a better accuracy
than using all of the samples
(\textbf{kmeans\_rarefaction\_fig\_5.pdf}). These results showed that
for modern sequencing depths, using the full datasets with rarefaction
and K-means clustering resulted in accuracies that were typically better
than those observed when removing the smallest 15\% of the samples from
each simulated dataset. When the original Figure 4 was recast with these
approaches, rarefaction performed at least as well as any of the other
tranformations with each distance calculation, except when used with the
Poisson distance (Figure \textbf{kmeans\_rarefaction15\_fig\_4.pdf}). It
is worth noting that at the largest effect sizes, K-means clustering did
not perform as well as PAM for some combinations of normalization method
and distance calculation (compare
\textbf{kmeans\_rarefaction15\_fig\_4.pdf} and
\textbf{pam\_subsample00\_fig\_4.pdf}); however, those combinations that
performed worse by K-means were not as good as rarefaction or
subsampling by either clustering method.

\hypertarget{the-distribution-of-library-sizes-was-not-typical-of-those-commonly-seen-in-microbiome-analyses}{%
\subsubsection{7. The distribution of library sizes was not typical of
those commonly seen in microbiome
analyses}\label{the-distribution-of-library-sizes-was-not-typical-of-those-commonly-seen-in-microbiome-analyses}}

As described above, the sequencing depths used in the 26 GlobalPatterns
datasets were used as the distribution to create sequencing depths for
the 80 samples that were generated in each simulation. The
GlobalPatterns datasets had a mean of 1085256.8 sequences and a median
of 1,106,849 sequences per dataset. The datasets ranged in sequencing
depth between 58,688 and 2,357,181 sequences for a 40.16-fold
difference. Rather than representing a typically observed distribution
of sequencing depths that would be skewed right, the sampling
distribution was normally distributed (Shapiro-Wilk test of normality,
P=0.57) (Figure \textbf{\emph{distribution\_shape.pdf}}). From these
simulations it is unclear how sensitive the various normalizations and
distance calculations were to a skewed distribution. A second limitation
of this sampling distribution is that it only contained 26 unique
sampling depths such that each sampling depth would have been re-used an
average of 3.08 times in each simulation. Yet, it is unlikely for a real
sequence collection to have duplicate sequencing depths. To reassess the
WNWN results in the context of a more typical distribution of sample
sizes, I created a new set of simulations to test the effect of the
shape of the distribution on the results. I created a simple sequencing
depth distribution where there were 80 depths logarithmically
distributed between the minimum and maximum sequencing depths of the
GlobalPatterns dataset (Figure \textbf{\emph{distribution\_shape.pdf}}).
The median of this distribution was 372,040 and the mean was 629824.8.
When I regenerated the original Figures 4 and 5 using the
log-distributed sequencing effort distribution, the differences in
normalization methods were more apparent (Figure
\textbf{\emph{fig\_4\_kmeans\_rarefaction00\_log.pdf}}). For each of the
distance calculators, rarefaction to the size of the smallest dataset
yielded accuracies that were at least as good as the other methods
across effect sizes and median sequencing depths. The difference was
most pronounced at smaller effect sizes and sequencing depths. When
comparing the performance of rarefaction across distance calculators for
different effect sizes, sequencing depths and size of smallest sample
(\textbf{\emph{fig\_5\_kmeans\_rarefaction\_log.pdf}}), the accuracies I
observed using the log-distributed sample sizes was at least as good as
those obtained using the GlobalPatterns-based distribution
\textbf{\emph{fig\_4\_kmeans\_rarefaction00\_a.pdf}}. The issue of the
number of samples in a study and the distribution of their sequencing
depths in the context of controlling for uneven sampling effort is
explored in far greater detail in an another analysis using sequencing
depths observed in actual biological samples {[}XXXXXX{]}.

\hypertarget{a-filtering-step-was-applied-to-remove-rare-taxa-from-the-simulated-datasets}{%
\subsubsection{8. A filtering step was applied to remove rare taxa from
the simulated
datasets}\label{a-filtering-step-was-applied-to-remove-rare-taxa-from-the-simulated-datasets}}

McMurdie and Holmes were emphatic that ``\textbf{rarefying biological
count data is statistically inadmissible} because it requires the
omission of available valid data'' (emphasis in original). Thus it is
strange that they argue against removing data when
rarefying/subsampling, but accept removing rare and low-prevalence OTUs
prior to normalizing their counts. This practice has become common in
microbiome studies and is the standard approach in tools such as dada2,
unoise, and deblur {[}XXXXXXXX{]}. However, my previous work has shown
that rare sequences from a poorly sequenced sample often appear in more
deeply sequened samples suggesting that they are not necessarily
artifacts. Furthermore, removing rare sequences alters the structure of
communities and has undesirable effects on downstream analyses
{[}XXXXXXXX{]}. Although my previous work does an extensive analysis of
the effects of removing rare sequences, I wanted to explore the effect
of filtering in the context of the WNWN simulation framework. For each
of the filtered and non-filtered OTU tables I calcualted the absolute
value of the difference in accuracy between each distance calculation
following the normalization procedure (Figure
\textbf{compare\_filter\_accuracy.pdf}). With the exception of the
Weighted UniFrac distances, each of the distance calculations and
normalization procedures were sensitive to the filtering. The
rarefaction data tended to be sensitive to filtering at small effect
sizes. Distances generated using raw counts, DESeq variance
stabilization, and Upper Quartile Log Fold Change tended to be more
sensitive to filtering at larger effect sizes. When using relative
abundance data, Bray-Curtis distances were sensitive to filtering at
small effect sizes and Unweighted UniFrac distances were sensitive at
large effect sizes. These trends appeared to be driven by the dependence
of the distance calculation on low abundance taxa. More surprising than
the effect of filtering on the mean absolute difference in clustering
accuracy was the wide variation in accuracies at each effect size. Among
the different normalization methods, the accuracies calculated using
rarefaction had the narrowest 95\% confidence interval for all distance
calculations except for calculating Unweighted UniFrac distances. For
these distances, using relative abundances had the narrowest range at
small effect sizes; rarefaction had the narrowest range at larger effect
sizes. Again, these trends appeared to be driven by the dependence on
low frequency taxa in Unweighted UniFrac, which is dependent on the
presence or absence of taxa rather than their abundance. Given my
previous work and the large variation caused by removing rare taxa, OTU
filtering should not be performed in microbiome analyses.

\hypertarget{no-accounting-for-difference-in-performance-when-library-sizes-are-confounded-with-treatment-group}{%
\subsubsection{9. No accounting for difference in performance when
library sizes are confounded with treatment
group}\label{no-accounting-for-difference-in-performance-when-library-sizes-are-confounded-with-treatment-group}}

In previous analyses I have observed that not using rarefaction can lead
to falsely detecing differences between communities when sampling effort
is confounded with the treatment group {[}XXXXXXX{]}. Such situations
have been observed when comparing communities at different body parts
where one site is more likely to generate contaminating sequence reads
from the host {[}XXXXXXX{]}. My previous analyses showed that
rarefaction did the best job of controlling the rates of false detection
(i.e., Type I errors) and maintaining the statistical power to detect
differences (i.e., 1-rate of Type II errors) of differences between
groups of samples. To determine whether this result was replicated with
the WNWN simulation framework, I created a skewed sampling distribution
using both the GlobalPatterns and Log-distributed sequence
distributions. To skew the sample counts the sequencing depth of samples
from one treatment group were drawn from below the median number of
sequences of the sampling distribution and those for the second
treatment group were from above the median. To assess Type I errors, I
compared the clustering accuracies using an effect size of 1.0 using
both the skewed and unskewed sampling distributions (Figure
\textbf{cluster\_skew\_compare\_i.pdf}). The samples should have only
been assigned to one cluster; however, each of the clustering methods
forced the samples into two clusters. So, when there are two groups of
40 samples that do not differ, the best a method could do would be to
correctly assign 41 of the 80 samples for an accuracy of 0.51. The Type
I error did not vary by method when the sequencing depth was not skewed.
Yet, when the sequencing depth was skewed, rarefaction was the most
consistent normalization method for controlling Type I errors. At larger
effect sizes the power to detect differences increased when the
sequencing depth was skewed (Figure
\textbf{cluster\_skew\_compare\_ii.pdf}). At the effect size of 1.15,
the rarefied data generated the highest accuracy clusters regardless of
whether the data were skewed. The exception to this were the Poisson
distance clusters, which generally clustered performed poorly. Although
the skewed simulation is extreme, it highlights the ability of
rarefaction to control Type I errors while maintaining high power and
the sensitivity of the other normalization methods.

\hypertarget{clustering-accuracy-was-used-rather-than-direct-comparisons-of-beta-diversity}{%
\subsubsection{10. Clustering accuracy was used rather than direct
comparisons of beta
diversity}\label{clustering-accuracy-was-used-rather-than-direct-comparisons-of-beta-diversity}}

Since WNWN was published, there has been controversy over the use of
clustering methods to group samples (i.e., enterotypes). Concerns have
been raised including whether such clustering should be done on
ecological distances or sequence counts and the biological
interpretation of such clusters {[}XXXXXXXXX{]}. As described in the
previous point, one notable challenge with using clustering accuracy as
the dependent variable is that the clustering methods force the samples
into one of two clusters. For the case where the effect size was 1.0, it
was impossible for all 80 samples to be assigned to a single cluster. As
has already been described in point 5, an additional problem with
clustering is the variation in the relative performance of a method
across conditions. A more commonly used approach for analyzing distance
matrices is to use a non-parametric analysis of variance test of the
various distance matrices (i.e., AMOVA, PERMANOVA,
NP-ANOVA){[}XXXXXXXXXXX{]}. I subjected each of the distance matrices to
such a test using \texttt{adonis2}, a function from the vegan R package
that implements this test to assess the effects of each normalization
and distance calculation method on the Type I errors and statistical
power. As was seen above, when sequencing depths were randomly
distributed across the two treatment groups, the Type I error did not
meaningfully deviate from the expected 5\% (Figure
\textbf{adonis\_skew\_compare\_i.pdf}). Again, when sequencing depths
were skewed between the two treatment groups, rarefaction was the only
normalization approach to control the Type I error. Similar to the
clustering accuracy results, when distances were calculated using
rarefaction, the tests consistently had the best statistical power
(Figure \textbf{adonis\_skew\_compare\_ii.pdf}). When considering both
Type I error and power, rarefaction performed the best among the
different normalizations.

\hypertarget{there-was-no-consideration-of-effects-of-normalization-methods-on-alpha-diversity-metrics}{%
\subsubsection{11. There was no consideration of effects of
normalization methods on alpha diversity
metrics}\label{there-was-no-consideration-of-effects-of-normalization-methods-on-alpha-diversity-metrics}}

Rarefaction was originally proposed as a method for controlling uneven
sampling effort when comparing community richness values
{[}XXXXXXXXXXXXXX{]}. Thus it was surprising that WNWN did not consider
the effect of the proposed normalizations on alpha-diversity metrics
such as richness or Shannon diversity. Therefore, for each of the
normalizations, I compared the richness and diversity of the two
treatment groups. The DESeq normalized data were not included because
the normalization produced negative values which were not compatible
with calculations of richness or Shannon diversity. Also, data from the
Upper Quartile Log Fold Change normalization were not used for richness
calculations since the normalization returned the same richness values
for each sample regardless of the treatment group. I assessed
significance for each iteration using the non-parametric Wilcoxon
two-sampled test. I compared the risk of committing Type I errors and
the power to detect differences by the different normalizations (Figure
\textbf{alpha\_compare.pdf}). For these analyses, I used the
GlobalPatterns data with the random and skewed distribution of samples.
Similar to the results in points 9 and 10, with the exception of
rarefaction, the simulations using a skewed distribution resulted in all
of the replicates having a significant test with each of the other
tranformations. The power to detect differences in richness and
diversity at effect sizes of 1.15 and greater with rarefaction was at
least as high as any of the other normalizations.

One odd result from this analysis was that the power to detect
differences in richness at small effect sizes increased between 1,000 to
2,000 sequences with values of 91.00 and 100.00, respectively. The power
then decreased with increasing sequencing effort to 57.00 with 50,000
sequences. This appeared to be because although the parent distributions
had very different shapes, they shared a large number of rare taxa. When
using rarefaction to compare the distributions at the size of the
smallest distribution (i.e., 3,598,077 sequences), the Feces parent
distribution had 1,559.65 OTUs and the Ocean had 1,335.00 OTUs; they
shared 894.66 OTUs. However, when comparing the distributions at 1,000
and 50,000 sequences, the Ocean distribution had greater richness than
the Feces distribution by 43.76 and 6.05, respectively. Therefore,
although more replicates at lower effect sizes yielded a small p-value
at shallow rather than deeper sequencing depths for richness, the
direction of the difference in richness was incorrect. This result
underscores the challenges of using presence-absence based-metrics like
richness and Jaccard and Unweighted UniFrac distances to compare
microbial communities.

\hypertarget{discussion}{%
\subsection{Discussion}\label{discussion}}

The conclusions from McMurdie and Holmes's study have had a lasting
impact on how researchers analyze microbiome sequence data. As I have
demonstrated using their original simulation approach their claims are
not supported. The most important points that lead to the difference in
our conclusions include the choice of clustering algorithm and
arbitrarily selecting a sequencing effort threshold that was used to
remove samples. Furthermore the decision to evaluate the effectiveness
of normalization methods based on clustering samples adds a layer of
analysis that has been controversial and not widely used. Ultimately,
the authors choice of the word ``rarefying'' has sewn confusion in the
field because it is often used in place of ``rarefaction''. As I have
demonstrated there were numerous choices throughout the orininal study
that made rarefying/rarefaction look worse than it truly was. In fact,
when the data are taken as a whole, rarefaction is the preferred
approach. Short of obtaining the exact same number of sequences from
each sample, rarefaction remains the best approach to control for uneven
sampling effort when analyzing alpha and beta diversity metrics.

Beyond the discussion of whether rarefaction is appropriate for
analyzing microbiome data it is worth commenting on McMurdie and
Holmes's advice to use DESeq's Variance Stabilization or edgeR's Upper
Quartile Log-fold Change normalization strategies. These methods have
been adopted from gene expression analysis to microbiome analysis. Gene
expression analysis implicitly assumes that all samples have the same
genes since. While this might work in comparing healthy and diseased
tissues from a cohort of patients, it does not generalize to those
patients' microbiota. Microbial populations are highly patchy in their
distribution. Thus, a zero count for gene expression is more likely to
represent a gene below the limit of detection whereas a zero count for a
microbiome analysis is more likely to represent the true absense of the
OTU. An example of where this is relevant is the necessity of adding a
pseudocount to all OTUs to perform both the edgeR and DESeq-based
normalizations (L443 and L487, respectively). In WNWN, a pseudocount of
1 was used. However, this value is arbitrary and the sensitivity of the
results can vary based on the patchiness of the communties being
analyzed. Since WNWN was published, compositional approaches have been
proposed to account for uneven sampling and to provide improved
interpretability {[}XXXXXX{]}. However, these methods also often require
the use of pseudocounts and are not actually insensitive to uneven
sampling {[}XXXXXXXX{]}. Rarefaction is preferred to these alternatives.

The choice of distance metric is a complicated question and the use of
six different metrics in WNWN illustrates the challenges. Within the
ecology literature, Euclidean distances are widely avoided because joint
absense is weighted the same as joint presence of taxa {[}XXXXXX{]}. As
discussed in point 11, metrics that are based on community membership
(i.e., the presence or absense of taxa) performed worse than those than
those that were based on community structure (i.e., their relative
abundance). For this reason, the Unweighted UniFrac and other metrics
like the Jaccard or Sorenson distance coefficients should likely be
avoided. The Poisson distance metric is largely novel to the microbial
ecology literature and performed no better than the more traditional
metrics. In the current analysis, the phylogenetic Weighted UniFrac
distance performed comparably to Bray-Cutis distances for clustering or
differentiating between communities with adonis. In practice, however,
the current reality is that it is computationally impractical to
construct phylogenetic trees with modern datasets. Although algorithms
have been developed to sidestep \emph{de novo} construction of trees
{[}XXXXXX{]}, these algorithms depend on reference-based clustering
strategies that have significant challenges {[}XXXXXX{]}. Among the
options analyzed here, Bray-Curtis is the most robust and practical
choice. Regardless, all of the methods perform best when using
rarefaction relative to the other normalization methods.

Although the authors claim that ``Rarefying counts requires an arbitrary
selection of a library size minimum that affects downstream inference''
(page 8, column 1, point 3), in acutal microbiome studies the selection
of a sampling depth is not as arbitrary as the authors claim. Rather, to
avoid ``p-hacking'', researchers pick a set of criteria where they will
include or exclude samples prior to testing their data. Examples of
criteria might include the presence of a large gap in the sequencing
effort distribution, a desire to include poorly sequenced controls, or
the \emph{a priori} stipulation of a minimum sequencing effort. To
mitigate concerns of arbitrary or engineered minimum library sizes,
researchers should indicate the rationale for the threshold they
selected.

WNWN performed a second set of simulations to address the effect of
normalization method on the ability to correctly detect differential
abundance of OTUs that were randomly selected to have their relative
abundances changed (i.e., Simulation B). Re-addresssing this set of
simulations is beyond the scope of the current anaylsis and others have
already contributed critiques {[}XXXXXXX{]}. However, many of the same
concerns addressed above would apply. Perhaps more importantly is the
fact that if the relative abundance of several OTUs increase, then the
relative abundance of all other OTUs would necessarily decrease. Thus,
one would expect every OTU to be differentially abundant. Because of
this, it is not possible to truly modify the abundance of a set of OTUs
indepdent of all othter OTUs. This is an important limitation of tests
of methods attempting to detect differentially abundant OTUs.
Regardless, a form of rarefaction could still be employed for detecting
differential abundance. One could subsmaple the data, perform the
statistical test, and identify the differentially abundant OTUs. This
process could be repeated. In my experience the largest differences
between subsamples are for low relative abundances OTUs, which are
unlikely to be biologically or statistically significant.

I am grateful to McMurdie and Holmes for providing the source code that
they used to conduct their analysis as R-flavored markdown files. This
provided me with a better understanding of their methods including
noticing that only a single subsampling step was performed for each
random seed and that only 5 random seeds were used. Furthermore, their
code enabled me to replicate and build upon their simulations. I have
used alternative simulation and evaluation strategies to look closer at
rarefaction and its alternatives and the practice of filtering low
abundance seqeunces. The results of those studies are similar to this
reanalysis. Rarefaction of unfiltered datasets yields the most robust
results.

\hypertarget{methods}{%
\subsection{Methods}\label{methods}}

\hypertarget{acknowledgements}{%
\subsection{Acknowledgements}\label{acknowledgements}}

\newpage

\hypertarget{references}{%
\subsection{References}\label{references}}

\setlength{\parindent}{-0.25in}
\setlength{\leftskip}{0.25in}

\noindent

\hypertarget{refs}{}
\begin{CSLReferences}{0}{0}
\end{CSLReferences}

\bibliography{ref}
\setlength{\parindent}{0in}
\setlength{\leftskip}{0in}

\newpage

\hypertarget{figures}{%
\subsection{Figures}\label{figures}}

\end{document}
